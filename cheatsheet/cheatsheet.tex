\documentclass{article}
\usepackage[a4paper, landscape]{geometry}
\usepackage{multicol}
\usepackage{amsmath,amssymb}
\usepackage{tikz}
\usetikzlibrary{decorations.pathmorphing,shapes.geometric}
\usepackage{xcolor}
\usepackage{enumitem}
\usepackage{colortbl}
\usepackage{array}
\usepackage{hyperref}
\usepackage{fontspec}

\setmainfont{Atkinson Hyperlegible}

% Page layout adjustments
\setlength{\topmargin}{-30mm}
\setlength{\textheight}{200mm}
\setlength{\textwidth}{287mm}
\setlength{\oddsidemargin}{-25mm}
\setlength{\evensidemargin}{-25mm}


% TikZ styles
\tikzstyle{mybox} = [draw=black, fill=white, very thick, rectangle, rounded corners, inner sep=10pt, inner ysep=10pt]
\tikzstyle{fancytitle} =[fill=black, text=white, font=\bfseries]

\begin{document}

\begin{center}{\huge{\textbf{LaTeX Cheatsheet Template}}}\end{center}
\vspace{1mm}
\raggedcolumns
\begin{multicols*}{3}

% Column 1: Lists, Tables, Math
% Box 1: List example
\begin{tikzpicture}
\node [mybox] (box){%
    \begin{minipage}{0.3\textwidth}
        \textbf{Example List:}
        \begin{itemize}[leftmargin=*]
            \item Point one
            \item Point two
            \item Point three
        \end{itemize}
    \end{minipage}
};
\node[fancytitle, right=10pt] at (box.north west) {List Example};
\end{tikzpicture}

% Box 2: Table example
\begin{tikzpicture}
\node [mybox] (box){%
    \begin{minipage}{0.3\textwidth}
        \textbf{Example Table:}
        \begin{center}
        \begin{tabular}{|c|c|c|}
        \hline
        Column 1 & Column 2 & Column 3 \\
        \hline
        A & B & C \\
        1 & 2 & 3 \\
        X & Y & Z \\
        \hline
        \end{tabular}
        \end{center}
    \end{minipage}
};
\node[fancytitle, right=10pt] at (box.north west) {Table Example};
\end{tikzpicture}

% Box 3: Math formula example
\begin{tikzpicture}
\node [mybox] (box){%
    \begin{minipage}{0.3\textwidth}
        \textbf{Math Formula:}
        \[ E = mc^2 \]
        Example of a displayed equation.
    \end{minipage}
};
\node[fancytitle, right=10pt] at (box.north west) {Math Example};
\end{tikzpicture}

\columnbreak

% Column 2: Hyperlinks, Graphics, Text Styling
% Box 4: Hyperlink example
\begin{tikzpicture}
\node [mybox] (box){%
    \begin{minipage}{0.3\textwidth}
        \textbf{Hyperlink Example:}
        \begin{itemize}[leftmargin=*]
            \item \href{https://www.latex-project.org}{LaTeX Project}
            \item \href{https://ctan.org}{CTAN Repository}
        \end{itemize}
    \end{minipage}
};
\node[fancytitle, right=10pt] at (box.north west) {Hyperlinks};
\end{tikzpicture}

% Box 5: Geometric shapes with TikZ
\begin{tikzpicture}
\node [mybox] (box){%
    \begin{minipage}{0.3\textwidth}
        \textbf{Geometric Shapes:}
        \begin{tikzpicture}[scale=0.6]
            \draw[fill=blue!30] (0,0) circle (1cm);
            \draw[fill=red!30] (2,0) rectangle (3,1);
            \draw[fill=green!30] (4,0) -- (5,1) -- (5,-1) -- cycle;
        \end{tikzpicture}
    \end{minipage}
};
\node[fancytitle, right=10pt] at (box.north west) {Shapes Example};
\end{tikzpicture}

% Box 6: Highlighted text
\begin{tikzpicture}
\node [mybox] (box){%
    \begin{minipage}{0.3\textwidth}
        \textbf{Highlighted Text:}
        \colorbox{yellow}{This is highlighted text.}
        \textbf{Bold text example.}
        \textit{Italic text example.}
    \end{minipage}
};
\node[fancytitle, right=10pt] at (box.north west) {Text Styling};
\end{tikzpicture}

\columnbreak

% Column 3: Enumerations, Diagrams, Colors
% Box 7: Enumerations
\begin{tikzpicture}
\node [mybox] (box){%
    \begin{minipage}{0.3\textwidth}
        \textbf{Enumeration Example:}
        \begin{enumerate}
            \item First item
            \item Second item
            \item Third item
        \end{enumerate}
    \end{minipage}
};
\node[fancytitle, right=10pt] at (box.north west) {Enumerations};
\end{tikzpicture}

% Box 8: Diagram example
\begin{tikzpicture}
\node [mybox] (box){%
    \begin{minipage}{0.3\textwidth}
        \textbf{Flow Diagram:}
        \begin{tikzpicture}[node distance=1.5cm, auto]
            \node[draw, circle] (start) {Start};
            \node[draw, rectangle, right of=start] (process) {Process};
            \node[draw, diamond, below of=process] (decision) {Decision};
            \draw[->] (start) -- (process);
            \draw[->] (process) -- (decision);
        \end{tikzpicture}
    \end{minipage}
};
\node[fancytitle, right=10pt] at (box.north west) {Flow Diagram};
\end{tikzpicture}

% Box 9: Color example
\begin{tikzpicture}
\node [mybox] (box){%
    \begin{minipage}{0.3\textwidth}
        \textbf{Color Example:}
        \textcolor{red}{This is red text.}
        \textcolor{blue}{This is blue text.}
        \textcolor{green}{This is green text.}
    \end{minipage}
};
\node[fancytitle, right=10pt] at (box.north west) {Colors};
\end{tikzpicture}

\end{multicols*}
\end{document}
